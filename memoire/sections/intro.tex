\chapter{Introduction}

% Présenter vraiment le truc dans sa globalité sans trop entrer dans les détails.
% Répondre aux questions : pourquoi ? en quoi ça consiste ? quel but ? …
% dire qu'on veut un truc centralisé, une grosse instance qui scale bien avec la possibilité de déployer un truc minimaliste à côté, etc...

\section{Le projet}

L'objectif de ce projet était de répondre à une série de problèmes rencontrés par l'organisation (fictive) Journalistes Sans Papiers.
% Ajouter JSP - Journalistes Sans Papiers au lexique des acronymes et abbréviations
JSP souhaitait un moyen le plus sécurisé possible de permettre à ses membres de communiquer les uns avec les autres.
La solution devait également supporter d'importants flux de données, et pouvoir fournir notamment un mode audio voire vidéo en plus du mode texte classique.
Enfin, l'organisation souhaitait la possibilité de déployer à la volée le service sur des réseaux isolés, permettant à ses membres d'utiliser la solution de communication dans des situations coupées du réseau Internet et des serveurs centraux de l'organisation.

\section{La solution}

Notre équipe a conçu la solution ChaTalK afin de répondre aux besoin de l'entreprise.
Il n'existait pas à notre connaissance de solution permettant les communication entièrement sécurisées de manière centralisée, dont un développement facile pour un néophyte puisse être possible.
ChaTalK est une application Web 2.0, dont l'infrastructure repose sur des cluster Kubernetes sur des machines Ubuntu Server, le côté serveur de l'application est un ensemble de services Go et PostgresQL, et le côté client est une interface écrite en React JS pour la partie navigateur et en Kotlin pour l'application mobile.
