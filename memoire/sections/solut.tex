\section{Solution}

La solution telle que nous l'avons conçue comprenait plusieurs éléments que nous allons rappeler ce-dessous.

\paragraph{Déploiement facile} Notre application devait pouvoir être facilement déployée par un néophyte, sur n'importe quelle machine de type Linux, rapidement et sans connexion Internet. Cette tâche était d'une priorité maximale%, nous ne l'avons toutefois pas menée à son terme.\\ commence peut être par un point positif?
%[[Expliquer]]

\paragraph{Sécurité} Notre application devait garantir la sécurité des flux de données, en particulier d'un point de vue intégrité et confidentialité. Seuls les envoyeurs et récepteurs des messages devaient avoir accès aux données.\\
Nous avons partiellement rempli cet objectif, des contraintes de temps liées à nos retards nous ayant forcé à abandonner le chiffrement des flux audiovisuels. L'authentification des utilisateurs a néanmoins été réalisée. Une couche de sécurité minimale basée sur SSL/TLS assure un niveau de sécurité décent, tant que les utilisateurs du service font confiance à l'administrateur des serveurs sur lesquels fonctionne la solution.

\paragraph{Tolérance aux pannes et passage à l'échelle} Grâce à son infrasctucture basée sur Kubernetes et à son architecture en micro-services, notre application est aisément capable de passer à l'échelle et de supporter la montée en charge, ainsi que de résister à la perte de machines dans les clusters. Les pires des cas entraînent des performances dégradées qui permettent de consulter les archives de conversation et donc de continuer une partie des activités des utilisateurs du service. % Phrase un peu bizarre je propose : "Dans les pires cas, les performances sont dégradées mais la consultation des archives de conversations est maintenue et on peut donc continuer une partie des activités des utilisateurs du service."

\paragraph{Temps réel} Notre application devait assurer des conversations en temps réel entre les utilisateurs.\\
Nos tests actuels ne sont pas représentatifs d'une utilisation réelle, mais montrent que la latence des communications textuelles sur notre solution est imperceptible à l'humain pour un petit nombre d'utilisateurs.

\paragraph{Interface utilisateur} Notre solution devait proposer une interface utilisateur compatible avec un maximum d'appareils.\\
Notre application frontale Web est utilisable sur tous les navigateurs modernes sans modification et sans distinction de système d'expoitation. De plus elle s'adapte à des tailles d'écran diverses et à des appareils de puissance variée. Bien que cette interface Web puisse être utilisée en tant qu'application mobile, une application mobile pour Android en Kotlin est également disponible, plus adaptée à l'utilisation sur des téléphones intelligents.\\ % tu utilises "cluster" plus haut mais pas smartphone ici?
L'interface utilisateur est épurée, présentant peu d'interactions possibles, documentée avec des icônographies habituelles telles qu'utilisées sur la majorité des autres services du même genre. % je propose "Nous avons choisi une interface utilisateur épurée, enconservant les icônographies habituelles afin de permettre une ergonomie maximale"

\paragraph{Modularité} Notre solution devait permettre à un individu déployant un serveur de choisir les capacités qu'il souhaitait donner à celui-ci.\\
À cause des contraintes de temps, et puisque la tâche n'était pas prioritaire, elle a été la première abandonnée totalement. Les fonctionnalités supplémentaires audio et vidéo ont été incluses directement dans l'application d'arrière-plan principale et le concept de modules serveur a été laissé à l'abandon.

\paragraph{Compatibilité et standards} Notre application devait être compatible avec un maximum de standards et ne pas utiliser de technologies trop peu orthodoxes.\\
Puisque l'interface utilisée est une interface Web, les quatre systèmes d'exploitation les plus courants, Microsoft Windows, Apple macOS, Apple iOS et Android, sont couverts tant qu'ils possèdent un navigateur Internet à jour. De plus, les systèmes d'exploitation GNU/Linux et BSD sont également couverts tant qu'ils possèdent un navigateur Internet à jour. Enfin, s'il n'existe pas d'application native pour iOS, il en existe une pour Android.\\
D'un point de vue des standards, notre application est accessible à des clients utilisant indifféremment de l'IPv4, de l'IPv6 ou les deux.

\paragraph{Internationalisation} Notre application se devait de supporter les principaux alphabets du monde et d'être disponible à minima en Anglais.\\
Puisque tous les textes que nous utilisons sont encodés en UTF-8, les alphabets divers sont gérés dans l'application. L'interface que nous avons développée l'a été en Anglais et est donc utilisable par des utilisateurs anglophones.
