\chapter{Conclusion}

En conclusion, nous avons rencontré des difficultés diverses dans la
réalisation du projet, ce qui nous a poussé à le modifier en conséquence.
Nous avons sélectionné les parties que nous jugions les plus critiques
afin qu’elles soient menées à bien pour garantir la fonctionnalité minimale
du projet. Ainsi notre solution effective ne respecte pas certains points
de la solution envisagée tels que: le déploiement rapide avec l’aide d’un
installateur de notre solution en version légère (prévue sans support de
la montée en charge; c’était plutôt une version de secours que JSP pouvait
être en mesure de déployer en cas de mission sur un lieu isolé d’Internet),
la sécurité sur l’audio/vidéo, la modularité de l’application sous forme de
greffons (bien que nous ayons conservé l’architecture en micro-service).

L’ensemble du groupe est plutôt d’accord sur le fait qu’il y avait une bonne
entente entre les différents membres et que le projet avait plutôt bien
commencé. Cependant suite à la recherche de stage ainsi que certains rendus
et examens dans d’autres matières, certains membres du groupe ont laissés le
projet en arrière-plan et ont eu des difficultés à pouvoir à nouveau retourner
sur le projet qui avait continué à avancer, hélas moins vite, sans eux.

Un certain manque de communication au sein du groupe pouvait se faire sentir,
empêchant une meilleure coordination. Cette dernière n’a été réellement
efficace que vers la fin du projet. Cependant ce projet a permi à chaque
membre de développer de nouvelles compétences qui ne pourront que leur être
bénéfiques et l’expérience accumulée sur ce projet permettra d’éviter que les
erreurs ne soient reproduites.

