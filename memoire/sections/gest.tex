\section{Gestion de projet}

% Décrire la gestion de projet
% Section Gestion de projet
%  - méthode de travail
%  - gestion et partage du travail
%  - outils utilisés
%  - réactivité, gestion humaine

\subsection{Organisation et commentaire général du projet}

Au départ du projet, nous avions prévu des binômes s'occupant
des tâches. Au fil du déroulement du projet, nous nous sommes
aperçus que certaines parties du projet avançaient plus vite
que d'autres, et que l'idée de binômes a changé pour confier
à chacun les tâches dans lesquelles il se sentait à l'aise.
\newline

Le suivi du projet était assuré par des réunions régulières
avec les clients / professeurs, entre une réunion par semaine
et une réunion pour deux semaines selon les périodes. Ces
réunions étaient suivies d'un compte-rendu à l'équipe projet,
durant lequel le nouveau planning était généralement présenté,
et où les tâches à venir étaient réparties entre les membres
de l'équipe.
\newline

Les membres de l'équipe étaient informés de leurs missions
via trois moyens principaux :
\begin{itemize}
  \item des \textit{issues} GitLab, avec une date, une mission et un responsable
  \item une notification écrite sur l'application de communication Discord
  \item une notification orale, généralement à la fin de la réunion de répartition des tâches
\end{itemize}
Malgré les efforts mis par le chef de projet pour notifier son
équipe des missions à réaliser, il est arrivé trop souvent
que les issues soient créées tardivement et que la
notification orale reste la seule information donnée à
l'équipe.
\newline

La plus grosse difficulté pour l'organisation de ce projet a
été de l'ordre de la communication. Gabriel a souvent été peu
réactif à officialiser par écrit les tâches à réaliser d'une
semaine sur l'autre. Le reste de l'équipe a été peu proactive,
en particulier au début, pour donner des retours sur ses
avancements et les tâches sur lesquelles elle travaillait et
comment ces tâches avançaient.

\subsection{Outils utilisés pour l'organisation et la gestion du projet}

\paragraph{Discord} L'outil principal utilisé pour
l'organisation et la gestion du projet a été le logiciel de
messagerie textuelle et vocale Discord. Nous l'avons choisi
car nous l'utilisons tous quotidiennement et qu'il nous
permet l'utilisation de certaines fonctionnalités pratiques
comme la création de canaux thématiques pour ne pas mélanger
les communication à propos de plusieurs tâches, ou l'épinglage
de messages pour les retrouver facilement.

\paragraph{Réunions en présentiel} Un outil très efficace que
nous avons utilisé lors de l'organisation du projet a été les
réunions en présentiel, avec autant de membres de l'équipe que
possible. Rétrospectivement, nous aurions du en faire plus
souvent. Lors de ces réunions, l'utilisation d'un tableau
blanc nous permettait de schématiser et de planifier. Ces
réunions étaient également propices à l'entraide et à la
pédagogie, permettant des explications en face à face direct.

\paragraph{Git et GitLab} Pour gérer le code source Git est
devenu un outil indispensable. Toutefois, outre cette
utilisation, la création de problèmes suivis a permis
de communiquer sur les spécification des tâches à plusieurs
reprises. Les historiques de \textit{commit} permettent plus
ou moins de voir qui a travaillé à quel moment et de détecter
les passages à vide.

\paragraph{Gantt Project} Pour créer et officialiser le
planning, l'outil Gantt Project a été utilisé, permettant de
créer simplement des diagrammes de Gantt. Ces diagrammes ont ensuite été partagés sur le dépôt GitLab pour que tous puissent les consulter.

\subsection{Réactivité et gestion humaine}

L'un des plus gros problèmes de ce projet a été la gestion de
la communication. D'un côté certains membres de l'équipe ont
été peu proactifs en ce qui concerne le fait d'informer les
autres de leur avancement, mais également de leurs
difficultés. Certains membres de l'équipe ont également été
peu réactifs à informer le chef du projet de leur
impossibilité temporaire à remplir leurs fonctions (maladie,
problèmes personnels grâves). D'un autre côté, le chef de
projet n'a pas été s'enquérir rapidement de l'état de ses
équipiers lorsque ceux-ci étaient absents ou ne donnaient pas
de nouvelles pendant plusieurs jours.
\newline

Ce manque de communication a posé problème pour la
redistribution des tâches auprès des autres membres de
l'équipe. Si l'informations avait été obtenue plus tôt, les
tâches non assurées auraient pu l'être par quelqu'un d'autre
et le retard aurait été mitigé.
\newline

Toutefois, lors des cinq dernières semaines du projet,
lorsqu'une tâche devait être réattribuée, elle l'a été sans
délai, pendant la période nécessaire.
