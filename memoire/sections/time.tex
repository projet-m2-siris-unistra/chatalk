\chapter{Emplois du temps et répartition du travail}

% Section Timesheets
%  - discussion sur la répartition du temps
%    et de la charge de travail
%  - présentation des six timesheets (timesheets à faire)

%\begin{figure}[h]
%  \caption{\label{lab} Title}
%  \includegraphics[width=15cm]{images/}
%\end{figure}

En tant que chef de projet, l'inexpérience de Gabriel
se voit également dans sa méconnaissance du concept de
\textit{timesheet} ou fiche horaire. Ce mécanisme permet
de garder une trace du temps passé par un individu sur
une tâche donnée.

N'ayant pas utilisé un logiciel dédié au suivi du temps
passé sur les tâches il est très compliqué d'estimer le
temps passé par chacun sur les tâches qui lui étaient
attribuées.

\subsection{Répartition du travail}

Dans l'équipe, les tâches ont lentement été réparties et peu de gens ont
réellement changé de fonction entre les début et la fin du projet.

\paragraph{Adrien Roberty} Adrien a d'abord été affecté la conception de
la sécurité de l'application. Plus tard dans le projet il a contribué à
l'écriture de certains services de l'arrière-plan de l'application.
Finalement il a conçu et programmé la partie communication audiovisuelle
de l'application.

\paragraph{Oussema Tourki} Oussema a été rapidement affecté à la programmation
du frontal de l'application en React JS, une technologie qu'il ne connaissait
pas et avec laquelle il beaucoup progressé depuis le début du projet. À
certains moments au milieu du projet Oussema semblait un peu embêté par ses
tâches, mais ne s'en ait jamais plaint.

\paragraph{Thomas Froeliger} Thomas a commencé le projet en travaillant en
tandem avec Adrien sur la conception de la sécurité de l'application, puis
a continué en développant en partie cette fonctionnalité. Il y a eu plusieurs
passages à vide dans le travail de Thomas, des périodes où aucune nouvelle
ne venaient de lui entre début et fin novembre en particulier.

\paragraph{Ludovic Muller} Ludovic a été un véritable pilier sur ce projet
capable de réaliser toute les tâches qu'on lui demandait, et parfois un peu
trop. Son influence a été à la fois bénéfique, car il permit de garder le
moral au mieux dans les passages difficiles, mais il semble que nous ayons
eu un peu trop tendance à nous appuyer sur lui, ce qui n'aurait peut-être pas
été le cas s'il avait été moins proactif. Il a toutefois réalisé
l'infrastructure de l'application avec brio.

\paragraph{Govindaraj Vetrivel} Govin est devenu responsable technique, car
il touchait à toutes les parties de l'application et en a rapidement compris
le fonctionnement malgré le manque de documentation, ce qui a été un atout
majeur pour aider le reste du groupe à faire fonctionner ensemble les divers
morceaux de l'application.

\paragraph{Gabriel Poittevin} Gabriel n'a pas été d'une grande aide technique,
s'étant surtout occupé de la base de données et de la gestion de projets.

